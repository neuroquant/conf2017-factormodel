Correlation estimation is a cornerstone of functional connectivity analyses across many neuroimaging modalities. Entries of a $p \times p$ correlation matrix correspond to pairwise correlations between neural activity two neural components such as anatomical regions or sensors or neurons. However, measurements of neural activity are often corrupted by modality specific artifacts. For instance, in resting or task-negative fMRI studies, the intensity of the BOLD signal varies by scanner and session. Furthermore, BOLD signals are easily corrupted by small movements of the participant, non-neural physiological signals such as respiratory and vascular changes. Such systematic measurement errors in the time-series in turn systematically bias sample correlations between brain regions. Fortunately, many studies also possess either measurements or estimates of such artifacts or confounds, particularly those related to motion and physiological noise. In this work, we introduce approximate factor models for correlation matrices that treat the observed correlation matrix as the sum of an artifact free correlation matrix and a nuisance correlation matrix. We illustrate two key benefits of explicitly replacing standard correlation estimation with factor model estimation. First, the signal energy in the nuisance correlation matrix yields an important quality control measure to diagnose whether functional connectivity estimates are corrupted by known artifacts. Second, by eliminating session specific artifacts from the original correlation matrix, we eliminate non-biological sources of between-session that affects cross-sectional and repeated measures studies of functional connectivity.